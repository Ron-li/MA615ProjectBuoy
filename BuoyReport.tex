\PassOptionsToPackage{unicode=true}{hyperref} % options for packages loaded elsewhere
\PassOptionsToPackage{hyphens}{url}
%
\documentclass[]{article}
\usepackage{lmodern}
\usepackage{amssymb,amsmath}
\usepackage{ifxetex,ifluatex}
\usepackage{fixltx2e} % provides \textsubscript
\ifnum 0\ifxetex 1\fi\ifluatex 1\fi=0 % if pdftex
  \usepackage[T1]{fontenc}
  \usepackage[utf8]{inputenc}
  \usepackage{textcomp} % provides euro and other symbols
\else % if luatex or xelatex
  \usepackage{unicode-math}
  \defaultfontfeatures{Ligatures=TeX,Scale=MatchLowercase}
\fi
% use upquote if available, for straight quotes in verbatim environments
\IfFileExists{upquote.sty}{\usepackage{upquote}}{}
% use microtype if available
\IfFileExists{microtype.sty}{%
\usepackage[]{microtype}
\UseMicrotypeSet[protrusion]{basicmath} % disable protrusion for tt fonts
}{}
\IfFileExists{parskip.sty}{%
\usepackage{parskip}
}{% else
\setlength{\parindent}{0pt}
\setlength{\parskip}{6pt plus 2pt minus 1pt}
}
\usepackage{hyperref}
\hypersetup{
            pdftitle={BuoyReport},
            pdfborder={0 0 0},
            breaklinks=true}
\urlstyle{same}  % don't use monospace font for urls
\usepackage[margin=1in]{geometry}
\usepackage{color}
\usepackage{fancyvrb}
\newcommand{\VerbBar}{|}
\newcommand{\VERB}{\Verb[commandchars=\\\{\}]}
\DefineVerbatimEnvironment{Highlighting}{Verbatim}{commandchars=\\\{\}}
% Add ',fontsize=\small' for more characters per line
\usepackage{framed}
\definecolor{shadecolor}{RGB}{248,248,248}
\newenvironment{Shaded}{\begin{snugshade}}{\end{snugshade}}
\newcommand{\AlertTok}[1]{\textcolor[rgb]{0.94,0.16,0.16}{#1}}
\newcommand{\AnnotationTok}[1]{\textcolor[rgb]{0.56,0.35,0.01}{\textbf{\textit{#1}}}}
\newcommand{\AttributeTok}[1]{\textcolor[rgb]{0.77,0.63,0.00}{#1}}
\newcommand{\BaseNTok}[1]{\textcolor[rgb]{0.00,0.00,0.81}{#1}}
\newcommand{\BuiltInTok}[1]{#1}
\newcommand{\CharTok}[1]{\textcolor[rgb]{0.31,0.60,0.02}{#1}}
\newcommand{\CommentTok}[1]{\textcolor[rgb]{0.56,0.35,0.01}{\textit{#1}}}
\newcommand{\CommentVarTok}[1]{\textcolor[rgb]{0.56,0.35,0.01}{\textbf{\textit{#1}}}}
\newcommand{\ConstantTok}[1]{\textcolor[rgb]{0.00,0.00,0.00}{#1}}
\newcommand{\ControlFlowTok}[1]{\textcolor[rgb]{0.13,0.29,0.53}{\textbf{#1}}}
\newcommand{\DataTypeTok}[1]{\textcolor[rgb]{0.13,0.29,0.53}{#1}}
\newcommand{\DecValTok}[1]{\textcolor[rgb]{0.00,0.00,0.81}{#1}}
\newcommand{\DocumentationTok}[1]{\textcolor[rgb]{0.56,0.35,0.01}{\textbf{\textit{#1}}}}
\newcommand{\ErrorTok}[1]{\textcolor[rgb]{0.64,0.00,0.00}{\textbf{#1}}}
\newcommand{\ExtensionTok}[1]{#1}
\newcommand{\FloatTok}[1]{\textcolor[rgb]{0.00,0.00,0.81}{#1}}
\newcommand{\FunctionTok}[1]{\textcolor[rgb]{0.00,0.00,0.00}{#1}}
\newcommand{\ImportTok}[1]{#1}
\newcommand{\InformationTok}[1]{\textcolor[rgb]{0.56,0.35,0.01}{\textbf{\textit{#1}}}}
\newcommand{\KeywordTok}[1]{\textcolor[rgb]{0.13,0.29,0.53}{\textbf{#1}}}
\newcommand{\NormalTok}[1]{#1}
\newcommand{\OperatorTok}[1]{\textcolor[rgb]{0.81,0.36,0.00}{\textbf{#1}}}
\newcommand{\OtherTok}[1]{\textcolor[rgb]{0.56,0.35,0.01}{#1}}
\newcommand{\PreprocessorTok}[1]{\textcolor[rgb]{0.56,0.35,0.01}{\textit{#1}}}
\newcommand{\RegionMarkerTok}[1]{#1}
\newcommand{\SpecialCharTok}[1]{\textcolor[rgb]{0.00,0.00,0.00}{#1}}
\newcommand{\SpecialStringTok}[1]{\textcolor[rgb]{0.31,0.60,0.02}{#1}}
\newcommand{\StringTok}[1]{\textcolor[rgb]{0.31,0.60,0.02}{#1}}
\newcommand{\VariableTok}[1]{\textcolor[rgb]{0.00,0.00,0.00}{#1}}
\newcommand{\VerbatimStringTok}[1]{\textcolor[rgb]{0.31,0.60,0.02}{#1}}
\newcommand{\WarningTok}[1]{\textcolor[rgb]{0.56,0.35,0.01}{\textbf{\textit{#1}}}}
\usepackage{graphicx,grffile}
\makeatletter
\def\maxwidth{\ifdim\Gin@nat@width>\linewidth\linewidth\else\Gin@nat@width\fi}
\def\maxheight{\ifdim\Gin@nat@height>\textheight\textheight\else\Gin@nat@height\fi}
\makeatother
% Scale images if necessary, so that they will not overflow the page
% margins by default, and it is still possible to overwrite the defaults
% using explicit options in \includegraphics[width, height, ...]{}
\setkeys{Gin}{width=\maxwidth,height=\maxheight,keepaspectratio}
\setlength{\emergencystretch}{3em}  % prevent overfull lines
\providecommand{\tightlist}{%
  \setlength{\itemsep}{0pt}\setlength{\parskip}{0pt}}
\setcounter{secnumdepth}{0}
% Redefines (sub)paragraphs to behave more like sections
\ifx\paragraph\undefined\else
\let\oldparagraph\paragraph
\renewcommand{\paragraph}[1]{\oldparagraph{#1}\mbox{}}
\fi
\ifx\subparagraph\undefined\else
\let\oldsubparagraph\subparagraph
\renewcommand{\subparagraph}[1]{\oldsubparagraph{#1}\mbox{}}
\fi

% set default figure placement to htbp
\makeatletter
\def\fps@figure{htbp}
\makeatother


\title{BuoyReport}
\author{}
\date{\vspace{-2.5em}}

\begin{document}
\maketitle

\hypertarget{rong-li-id-u73933267}{%
\paragraph{Rong Li, id U73933267}\label{rong-li-id-u73933267}}

\hypertarget{understanding-of-the-question}{%
\subsection{Understanding of the
question}\label{understanding-of-the-question}}

We need to analysis 20-years of data collected by a single weather buoy
in the NOAA National Data Buoy Center. The raw data concludes date, wind
direction, wave height, air temperature(ATMP), water temperature(WTMP),
dew point and so on. What we should do is mining the information from
this data and find out whether there is an evidence of global warming.

\hypertarget{my-approach}{%
\subsection{My approach}\label{my-approach}}

First, I use R to clean and select the raw data. Since temperature is
the most important signature of globle warming, I include the air
temperature(ATMP) and water temprature(WTMP) at first. Then I consider
which of the other columns related to temperature. Obviously, the
seasonal change shouldn't be neglected. So the dates are include, too.\\
Secondly, I use R to analysis the data. I do some exploratory research
to get familiar with this data,but there is no obvious trend. So I use
the ARIMA model in time series to analysis the data. The function
auto.arima() helps me to find appropriate peremeters rapidly. After I
test the model, I can draw the conclusion.

\hypertarget{how-i-organized-the-work}{%
\subsection{How I organized the work}\label{how-i-organized-the-work}}

Part1: Processing the data. In the Raw\_data.R, I put the code about
processing the raw data in it.\\
Part2: Data analysis. The data\_analysis.R includes the exploratory
research and time series analysis.

\hypertarget{part1-processing-raw-data}{%
\subsubsection{Part1: Processing raw
data}\label{part1-processing-raw-data}}

In this part, I access to the data of 1999-2018 online. Then I select
ATMP, WTMP, date, and clean up the abnormal data. Finally I put the
annual data together, turn them into a dataframe which I will use later.

\hypertarget{the-packages-i-use}{%
\paragraph{The packages I use:}\label{the-packages-i-use}}

\begin{Shaded}
\begin{Highlighting}[]
\KeywordTok{library}\NormalTok{(tidyverse)}
\end{Highlighting}
\end{Shaded}

\begin{verbatim}
## -- Attaching packages ----------------------------------------- tidyverse 1.3.0 --
\end{verbatim}

\begin{verbatim}
## v ggplot2 3.3.2     v purrr   0.3.4
## v tibble  3.0.3     v dplyr   1.0.2
## v tidyr   1.1.2     v stringr 1.4.0
## v readr   1.3.1     v forcats 0.5.0
\end{verbatim}

\begin{verbatim}
## -- Conflicts -------------------------------------------- tidyverse_conflicts() --
## x dplyr::filter() masks stats::filter()
## x dplyr::lag()    masks stats::lag()
\end{verbatim}

\begin{Shaded}
\begin{Highlighting}[]
\KeywordTok{library}\NormalTok{(stringr)}
\KeywordTok{library}\NormalTok{(ggplot2)}
\KeywordTok{library}\NormalTok{(lubridate)}
\end{Highlighting}
\end{Shaded}

\begin{verbatim}
## 
## Attaching package: 'lubridate'
\end{verbatim}

\begin{verbatim}
## The following objects are masked from 'package:base':
## 
##     date, intersect, setdiff, union
\end{verbatim}

\begin{Shaded}
\begin{Highlighting}[]
\KeywordTok{library}\NormalTok{(tseries)}
\end{Highlighting}
\end{Shaded}

\begin{verbatim}
## Registered S3 method overwritten by 'quantmod':
##   method            from
##   as.zoo.data.frame zoo
\end{verbatim}

\begin{Shaded}
\begin{Highlighting}[]
\KeywordTok{library}\NormalTok{(forecast)}
\KeywordTok{library}\NormalTok{(urca)}
\end{Highlighting}
\end{Shaded}

\hypertarget{download-the-data-of-1999-2018-online-and-save-them-as-mr1999-mr2018.-change-some-files-column-names-to-keep-consistent.}{%
\paragraph{Download the data of 1999-2018 online and save them as
``mr1999-mr2018''. Change some files' column names to keep
consistent.}\label{download-the-data-of-1999-2018-online-and-save-them-as-mr1999-mr2018.-change-some-files-column-names-to-keep-consistent.}}

\begin{Shaded}
\begin{Highlighting}[]
\NormalTok{url_}\DecValTok{1}\NormalTok{ <-}\StringTok{ "http://www.ndbc.noaa.gov/view_text_file.php?filename=mlrf1h"}
\NormalTok{url_}\DecValTok{2}\NormalTok{ <-}\StringTok{ ".txt.gz&dir=data/historical/stdmet/"}
\NormalTok{years <-}\StringTok{ }\KeywordTok{c}\NormalTok{(}\DecValTok{1999}\OperatorTok{:}\DecValTok{2018}\NormalTok{)}
\NormalTok{urls <-}\StringTok{ }\KeywordTok{str_c}\NormalTok{(url_}\DecValTok{1}\NormalTok{, years, url_}\DecValTok{2}\NormalTok{, }\DataTypeTok{sep =} \StringTok{""}\NormalTok{)}
\NormalTok{filenames <-}\StringTok{ }\KeywordTok{str_c}\NormalTok{(}\StringTok{"mr"}\NormalTok{, years, }\DataTypeTok{sep =} \StringTok{""}\NormalTok{)}


\CommentTok{# Year 1999 - 2006}
\ControlFlowTok{for}\NormalTok{(i }\ControlFlowTok{in} \DecValTok{1}\OperatorTok{:}\DecValTok{8}\NormalTok{)\{}
  \KeywordTok{suppressMessages}\NormalTok{(}
    \CommentTok{# Fill any missing values with NA:}
    \KeywordTok{assign}\NormalTok{(filenames[i], }\KeywordTok{read.table}\NormalTok{(urls[i], }\DataTypeTok{header =} \OtherTok{TRUE}\NormalTok{, }\DataTypeTok{fill =} \OtherTok{TRUE}\NormalTok{))}
\NormalTok{  )}
  
\NormalTok{\}}
\CommentTok{# Year 2007 - 2018}
\ControlFlowTok{for}\NormalTok{(i }\ControlFlowTok{in} \DecValTok{9}\OperatorTok{:}\DecValTok{20}\NormalTok{)\{}
  \KeywordTok{suppressMessages}\NormalTok{(}
    \CommentTok{# Fill any missing values with NA and use the same column names as year 2006}
    \KeywordTok{assign}\NormalTok{(filenames[i], }\KeywordTok{read.table}\NormalTok{(urls[i], }\DataTypeTok{header =} \OtherTok{FALSE}\NormalTok{, }
                                    \DataTypeTok{fill =} \OtherTok{TRUE}\NormalTok{, }\DataTypeTok{col.names =} \KeywordTok{colnames}\NormalTok{(mr2006))),}
\NormalTok{  )}
  
\NormalTok{\}}
\end{Highlighting}
\end{Shaded}

\hypertarget{screen-out-data-at-13-oclock.}{%
\paragraph{Screen out data at 13
o'clock.}\label{screen-out-data-at-13-oclock.}}

\begin{Shaded}
\begin{Highlighting}[]
\NormalTok{i<-}\DecValTok{1999}
\ControlFlowTok{repeat}\NormalTok{ \{ }
  \KeywordTok{assign}\NormalTok{(}\KeywordTok{paste}\NormalTok{(}\StringTok{"mr"}\NormalTok{,}\KeywordTok{as.character}\NormalTok{(i),}\DataTypeTok{sep=}\StringTok{""}\NormalTok{),}\KeywordTok{get}\NormalTok{(}\KeywordTok{paste}\NormalTok{(}\StringTok{"mr"}\NormalTok{,}\KeywordTok{as.character}\NormalTok{(i),}\DataTypeTok{sep=}\StringTok{""}\NormalTok{))[}\KeywordTok{which}\NormalTok{(}\KeywordTok{get}\NormalTok{(}\KeywordTok{paste}\NormalTok{(}\StringTok{"mr"}\NormalTok{,}\KeywordTok{as.character}\NormalTok{(i),}\DataTypeTok{sep=}\StringTok{""}\NormalTok{))}\OperatorTok{$}\NormalTok{hh }\OperatorTok{==}\StringTok{ }\DecValTok{13}\NormalTok{), ])}
\NormalTok{  i=i}\OperatorTok{+}\DecValTok{1}
  \ControlFlowTok{if}\NormalTok{(i}\OperatorTok{>}\DecValTok{2018}\NormalTok{)}
\NormalTok{  \{}\ControlFlowTok{break}\NormalTok{\}}
\NormalTok{\}}
\end{Highlighting}
\end{Shaded}

\hypertarget{put-the-data-together-into-a-dataframe-called-mrc-and-select-yyyy-mm-dd-atmp-wtmp.}{%
\paragraph{Put the data together into a dataframe called MRC and select
``YYYY'', ``MM'', ``DD'', ``ATMP'',
``WTMP''.}\label{put-the-data-together-into-a-dataframe-called-mrc-and-select-yyyy-mm-dd-atmp-wtmp.}}

\begin{Shaded}
\begin{Highlighting}[]
\NormalTok{mr1999}\OperatorTok{$}\NormalTok{TIDE <-}\StringTok{ }\OtherTok{NA}
\NormalTok{n <-}\StringTok{ }\DecValTok{20}
\ControlFlowTok{for}\NormalTok{ (i }\ControlFlowTok{in} \DecValTok{1}\OperatorTok{:}\NormalTok{n)\{}
\NormalTok{  file <-}\StringTok{ }\KeywordTok{get}\NormalTok{(filenames[i])}
  \KeywordTok{colnames}\NormalTok{(file)[}\DecValTok{1}\NormalTok{] <-}\StringTok{"YYYY"}
  \ControlFlowTok{if}\NormalTok{(}\KeywordTok{ncol}\NormalTok{(file) }\OperatorTok{==}\StringTok{ }\DecValTok{18}\NormalTok{)\{}
\NormalTok{    file <-}\StringTok{ }\KeywordTok{subset}\NormalTok{(file, }\DataTypeTok{select =} \OperatorTok{-}\NormalTok{mm )}
\NormalTok{  \}}
  \ControlFlowTok{if}\NormalTok{(i }\OperatorTok{==}\StringTok{ }\DecValTok{1}\NormalTok{)\{}
\NormalTok{    MRC <-}\StringTok{ }\NormalTok{file}
\NormalTok{  \}}\ControlFlowTok{else}\NormalTok{\{}
\NormalTok{    MRC <-}\StringTok{ }\KeywordTok{rbind.data.frame}\NormalTok{(MRC, file)}
\NormalTok{  \}}
  
\NormalTok{\}}

\NormalTok{MRC<-MRC[}\KeywordTok{c}\NormalTok{(}\DecValTok{1}\NormalTok{,}\DecValTok{2}\NormalTok{,}\DecValTok{3}\NormalTok{,}\DecValTok{13}\NormalTok{,}\DecValTok{14}\NormalTok{)]}
\end{Highlighting}
\end{Shaded}

\hypertarget{delete-the-abnormal-data.}{%
\paragraph{Delete the abnormal data.}\label{delete-the-abnormal-data.}}

\begin{Shaded}
\begin{Highlighting}[]
\NormalTok{MRC}\OperatorTok{$}\NormalTok{ATMP[}\KeywordTok{which}\NormalTok{(MRC}\OperatorTok{$}\NormalTok{ATMP}\OperatorTok{>=}\DecValTok{100}\NormalTok{)]=}\OtherTok{NA}
\NormalTok{MRC}\OperatorTok{$}\NormalTok{WTMP[}\KeywordTok{which}\NormalTok{(MRC}\OperatorTok{$}\NormalTok{WTMP}\OperatorTok{>=}\DecValTok{100}\NormalTok{)]=}\OtherTok{NA}
\NormalTok{MRC=}\KeywordTok{na.omit}\NormalTok{(MRC)}
\end{Highlighting}
\end{Shaded}

At the end of the part, we have obtained a dataframe called MRC which we
can use later.

\hypertarget{part2-data-analysis}{%
\subsubsection{Part2: Data analysis}\label{part2-data-analysis}}

In this part, I use MRC to further my research. I do some exploratory
research such as ploting the boxplot of ATMP versus year and the boxplot
of WTMP versus year to get a general understanding of the data. I also
go deeper in it and change the data into time series. After testing the
ARIMA model I use, I draw a conclusion that there is no obvious evidence
of global warming.

\hypertarget{boxplot}{%
\paragraph{Boxplot}\label{boxplot}}

\begin{Shaded}
\begin{Highlighting}[]
\NormalTok{MRC}\OperatorTok{$}\NormalTok{factor <-}\StringTok{ }\KeywordTok{as.factor}\NormalTok{(MRC}\OperatorTok{$}\NormalTok{YYYY)}

\CommentTok{#Plot the boxplot of ATMP versus Year}
\KeywordTok{ggplot}\NormalTok{(MRC, }\KeywordTok{aes}\NormalTok{(}\DataTypeTok{x =}\NormalTok{ factor, }\DataTypeTok{y =}\NormalTok{ ATMP)) }\OperatorTok{+}
\StringTok{  }\KeywordTok{geom_boxplot}\NormalTok{(}\DataTypeTok{alpha=}\FloatTok{0.7}\NormalTok{) }\OperatorTok{+}
\StringTok{  }\KeywordTok{scale_y_continuous}\NormalTok{(}\DataTypeTok{name =} \StringTok{"ATMP"}\NormalTok{)}\OperatorTok{+}
\StringTok{  }\KeywordTok{scale_x_discrete}\NormalTok{(}\DataTypeTok{name =} \StringTok{"YEAR"}\NormalTok{) }\OperatorTok{+}
\StringTok{  }\KeywordTok{ggtitle}\NormalTok{(}\StringTok{"Boxplot of ATMP"}\NormalTok{)}
\end{Highlighting}
\end{Shaded}

\includegraphics{BuoyReport_files/figure-latex/unnamed-chunk-6-1.pdf}

\begin{Shaded}
\begin{Highlighting}[]
\CommentTok{#Plot the boxplot of WTMP versus Year}
\KeywordTok{ggplot}\NormalTok{(MRC, }\KeywordTok{aes}\NormalTok{(}\DataTypeTok{x =}\NormalTok{ factor, }\DataTypeTok{y =}\NormalTok{ WTMP)) }\OperatorTok{+}
\StringTok{  }\KeywordTok{geom_boxplot}\NormalTok{(}\DataTypeTok{alpha=}\FloatTok{0.7}\NormalTok{) }\OperatorTok{+}
\StringTok{  }\KeywordTok{scale_y_continuous}\NormalTok{(}\DataTypeTok{name =} \StringTok{"WTMP"}\NormalTok{)}\OperatorTok{+}
\StringTok{  }\KeywordTok{scale_x_discrete}\NormalTok{(}\DataTypeTok{name =} \StringTok{"YEAR"}\NormalTok{) }\OperatorTok{+}
\StringTok{  }\KeywordTok{ggtitle}\NormalTok{(}\StringTok{"Boxplot of WTMP"}\NormalTok{)}
\end{Highlighting}
\end{Shaded}

\includegraphics{BuoyReport_files/figure-latex/unnamed-chunk-6-2.pdf}

From the picture above, we can't draw conclusions because there is no
obvious trend. So we need to go deeper.

\hypertarget{calculate-the-monthly-average-of-atmpwtmp-every-year-and-form-time-series}{%
\paragraph{Calculate the monthly average of ATMP\&WTMP every year and
form time
series}\label{calculate-the-monthly-average-of-atmpwtmp-every-year-and-form-time-series}}

\begin{Shaded}
\begin{Highlighting}[]
\NormalTok{ameans <-}\StringTok{ }\KeywordTok{array}\NormalTok{()}
\NormalTok{wmeans <-}\StringTok{ }\KeywordTok{array}\NormalTok{()}
\NormalTok{z <-}\StringTok{ }\DecValTok{1}
\ControlFlowTok{for}\NormalTok{ (i }\ControlFlowTok{in} \DecValTok{1999}\OperatorTok{:}\DecValTok{2018}\NormalTok{)\{}
  \ControlFlowTok{for}\NormalTok{ (j }\ControlFlowTok{in} \DecValTok{1}\OperatorTok{:}\DecValTok{12}\NormalTok{)\{}
\NormalTok{    ameans[z] <-}\StringTok{ }\KeywordTok{mean}\NormalTok{(}\KeywordTok{subset}\NormalTok{(MRC, YYYY }\OperatorTok{==}\StringTok{ }\NormalTok{i)}\OperatorTok{$}\NormalTok{ATMP[}\KeywordTok{which}\NormalTok{(}\KeywordTok{subset}\NormalTok{(MRC, YYYY }\OperatorTok{==}\StringTok{ }\NormalTok{i)}\OperatorTok{$}\NormalTok{MM }\OperatorTok{==}\StringTok{ }\NormalTok{j)])}
\NormalTok{    wmeans[z] <-}\StringTok{ }\KeywordTok{mean}\NormalTok{(}\KeywordTok{subset}\NormalTok{(MRC, YYYY }\OperatorTok{==}\StringTok{ }\NormalTok{i)}\OperatorTok{$}\NormalTok{WTMP[}\KeywordTok{which}\NormalTok{(}\KeywordTok{subset}\NormalTok{(MRC, YYYY }\OperatorTok{==}\StringTok{ }\NormalTok{i)}\OperatorTok{$}\NormalTok{MM }\OperatorTok{==}\StringTok{ }\NormalTok{j)])}
\NormalTok{    z <-}\StringTok{ }\NormalTok{z}\OperatorTok{+}\DecValTok{1}
    \ControlFlowTok{if}\NormalTok{ (z }\OperatorTok{>}\StringTok{ }\DecValTok{5}\NormalTok{)\{  }
      \ControlFlowTok{if}\NormalTok{ (}\KeywordTok{is.na}\NormalTok{(ameans[z}\DecValTok{-2}\NormalTok{]))\{}
\NormalTok{        ameans[z}\DecValTok{-2}\NormalTok{] <-}\StringTok{ }\NormalTok{(ameans[z}\DecValTok{-3}\NormalTok{]}\OperatorTok{+}\NormalTok{ameans[z}\DecValTok{-1}\NormalTok{])}\OperatorTok{/}\DecValTok{2} 
\NormalTok{      \}}
      \ControlFlowTok{if}\NormalTok{ (}\KeywordTok{is.na}\NormalTok{(wmeans[z}\DecValTok{-2}\NormalTok{]))\{}
\NormalTok{        wmeans[z}\DecValTok{-2}\NormalTok{] <-}\StringTok{ }\NormalTok{(wmeans[z}\DecValTok{-3}\NormalTok{]}\OperatorTok{+}\NormalTok{wmeans[z}\DecValTok{-1}\NormalTok{])}\OperatorTok{/}\DecValTok{2} 
\NormalTok{      \}}
\NormalTok{    \}}
\NormalTok{  \}}
\NormalTok{\}}
\NormalTok{atmpseries <-}\StringTok{ }\KeywordTok{ts}\NormalTok{(ameans, }\DataTypeTok{frequency=}\DecValTok{12}\NormalTok{, }\DataTypeTok{start=}\KeywordTok{c}\NormalTok{(}\DecValTok{1999}\NormalTok{,}\DecValTok{1}\NormalTok{))}
\NormalTok{wtmpseries <-}\StringTok{ }\KeywordTok{ts}\NormalTok{(wmeans, }\DataTypeTok{frequency=}\DecValTok{12}\NormalTok{, }\DataTypeTok{start=}\KeywordTok{c}\NormalTok{(}\DecValTok{1999}\NormalTok{,}\DecValTok{1}\NormalTok{)) }
\KeywordTok{plot}\NormalTok{(atmpseries, }\DataTypeTok{main =} \StringTok{"The Time Series of ATMP"}\NormalTok{, }\DataTypeTok{xlab =} \StringTok{"Date"}\NormalTok{)}
\end{Highlighting}
\end{Shaded}

\includegraphics{BuoyReport_files/figure-latex/unnamed-chunk-7-1.pdf}

\begin{Shaded}
\begin{Highlighting}[]
\KeywordTok{plot}\NormalTok{(wtmpseries, }\DataTypeTok{main =} \StringTok{"The Time Series of WTMP"}\NormalTok{, }\DataTypeTok{xlab =} \StringTok{"Date"}\NormalTok{)}
\end{Highlighting}
\end{Shaded}

\includegraphics{BuoyReport_files/figure-latex/unnamed-chunk-7-2.pdf}
The pictures indicate that there should be seasonal trend in the
temperature. Since the two pictures are very alike which means the
trends of ATMP and WTMP are very similar, I only need to analysis one of
them. In this case, I choose to analysis the ATMP.

\#\#\#\#Display the time series of ATMP.

\begin{Shaded}
\begin{Highlighting}[]
\KeywordTok{tsdisplay}\NormalTok{(atmpseries)}
\end{Highlighting}
\end{Shaded}

\includegraphics{BuoyReport_files/figure-latex/unnamed-chunk-8-1.pdf}

\hypertarget{decompose-the-time-series}{%
\paragraph{Decompose the time series}\label{decompose-the-time-series}}

\begin{Shaded}
\begin{Highlighting}[]
\NormalTok{dca <-}\StringTok{ }\KeywordTok{decompose}\NormalTok{(atmpseries)}
\KeywordTok{plot}\NormalTok{(dca)}
\end{Highlighting}
\end{Shaded}

\includegraphics{BuoyReport_files/figure-latex/unnamed-chunk-9-1.pdf}

In the graph, the seasonal effection is apperant and the trend is
unclear.

\hypertarget{find-out-the-seasonal-trend-of-the-data.}{%
\paragraph{Find out the seasonal trend of the
data.}\label{find-out-the-seasonal-trend-of-the-data.}}

\begin{Shaded}
\begin{Highlighting}[]
\NormalTok{seasona<-dca}\OperatorTok{$}\NormalTok{figure}
\KeywordTok{plot}\NormalTok{(seasona,}\DataTypeTok{type =} \StringTok{"b"}\NormalTok{,}\DataTypeTok{xaxt=}\StringTok{"n"}\NormalTok{,}\DataTypeTok{xlab =} \StringTok{""}\NormalTok{)}
\end{Highlighting}
\end{Shaded}

\includegraphics{BuoyReport_files/figure-latex/unnamed-chunk-10-1.pdf}

\hypertarget{use-function-auto.arima-to-get-the-peremeter-of-arima-model.}{%
\paragraph{Use function auto.arima() to get the peremeter of ARIMA
model.}\label{use-function-auto.arima-to-get-the-peremeter-of-arima-model.}}

\begin{Shaded}
\begin{Highlighting}[]
\NormalTok{a<-}\KeywordTok{auto.arima}\NormalTok{(atmpseries)}
\KeywordTok{print}\NormalTok{(a)}
\end{Highlighting}
\end{Shaded}

\begin{verbatim}
## Series: atmpseries 
## ARIMA(0,0,1)(2,1,0)[12] with drift 
## 
## Coefficients:
##          ma1     sar1     sar2   drift
##       0.2279  -0.5206  -0.2393  0.0019
## s.e.  0.0636   0.0657   0.0667  0.0055
## 
## sigma^2 estimated as 1.941:  log likelihood=-398.99
## AIC=807.99   AICc=808.26   BIC=825.14
\end{verbatim}

The model is ARIMA(0, 0, 1)x(2, 1, 0).

\hypertarget{fit-the-model-and-plot-the-predicted.}{%
\paragraph{Fit the model and plot the
predicted.}\label{fit-the-model-and-plot-the-predicted.}}

\begin{Shaded}
\begin{Highlighting}[]
\NormalTok{fit<-}\KeywordTok{arima}\NormalTok{(atmpseries,}\DataTypeTok{order =} \KeywordTok{c}\NormalTok{(}\DecValTok{0}\NormalTok{, }\DecValTok{0}\NormalTok{, }\DecValTok{1}\NormalTok{),}\DataTypeTok{seasonal =} \KeywordTok{list}\NormalTok{(}\DataTypeTok{order=}\KeywordTok{c}\NormalTok{(}\DecValTok{2}\NormalTok{, }\DecValTok{1}\NormalTok{, }\DecValTok{0}\NormalTok{),}\DataTypeTok{period=}\DecValTok{12}\NormalTok{))}
\KeywordTok{print}\NormalTok{(fit)}
\end{Highlighting}
\end{Shaded}

\begin{verbatim}
## 
## Call:
## arima(x = atmpseries, order = c(0, 0, 1), seasonal = list(order = c(2, 1, 0), 
##     period = 12))
## 
## Coefficients:
##          ma1     sar1     sar2
##       0.2286  -0.5200  -0.2386
## s.e.  0.0635   0.0657   0.0667
## 
## sigma^2 estimated as 1.908:  log likelihood = -399.05,  aic = 806.11
\end{verbatim}

\begin{Shaded}
\begin{Highlighting}[]
\NormalTok{fore=}\KeywordTok{predict}\NormalTok{(fit)}
\KeywordTok{ts.plot}\NormalTok{(atmpseries, fore}\OperatorTok{$}\NormalTok{pred, }\DataTypeTok{col=}\KeywordTok{c}\NormalTok{(}\DecValTok{1}\NormalTok{,}\DecValTok{2}\NormalTok{,}\DecValTok{4}\NormalTok{,}\DecValTok{4}\NormalTok{), }\DataTypeTok{lty=}\KeywordTok{c}\NormalTok{(}\DecValTok{1}\NormalTok{,}\DecValTok{1}\NormalTok{,}\DecValTok{2}\NormalTok{,}\DecValTok{2}\NormalTok{))}
\end{Highlighting}
\end{Shaded}

\includegraphics{BuoyReport_files/figure-latex/unnamed-chunk-12-1.pdf}

\hypertarget{test-the-model}{%
\paragraph{Test the model}\label{test-the-model}}

\begin{Shaded}
\begin{Highlighting}[]
\KeywordTok{ur.df}\NormalTok{(atmpseries)}
\end{Highlighting}
\end{Shaded}

\begin{verbatim}
## 
## ############################################################### 
## # Augmented Dickey-Fuller Test Unit Root / Cointegration Test # 
## ############################################################### 
## 
## The value of the test statistic is: -0.8381
\end{verbatim}

The value of the test statistic is -0.8381 \textless{} 5\%, which refers
the series is stable.\\
We can draw the conclusion that there is no obvious evidence of global
warming.

\hypertarget{my-conlusions}{%
\subsection{My conlusions}\label{my-conlusions}}

Through exploratory research and time series analysis, there is no
conclusive evidence of global warming in the data.

\hypertarget{my-reference}{%
\subsection{My reference}\label{my-reference}}

{[}data
scource{]}\url{http://www.ndbc.noaa.gov/view_text_file.php?filename=mlrf1h1999.txt.gz\&dir=data/historical/stdmet/}\\
{[}Methods in time
series{]}\url{https://blog.csdn.net/jiabiao1602/article/details/43153139}\\
{[}R
packages{]}\url{https://cran.r-project.org/web/packages/citation/index.html}

\end{document}
